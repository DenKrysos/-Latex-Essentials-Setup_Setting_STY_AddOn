%________________________________________________________________________
%------------------------------------------------------------------------
%						Dual Entrys
%/\/\/\/\/\/\/\/\/\/\/\/\/\/\/\/\/\/\/\/\/\/\/\/\/\/\/\/\/\/\/\/\/\/\/\/\
% New Makro, to easily create a DualEntry for simultaneous definition of Acronym \& Glossary Entry, with reference
% -- Version 1 (less powerful. If possible, use Version 2):
% \newcommand*{\newdualentry}[5][]{%
% 	\newglossaryentry{gls-#2}{name={#4},text={#3\glsadd{#2}},%
% 		description={#5},#1%
% 	}%
% 	\newacronym{#2}{#3\glsadd{gls-#2}}{#4}%
% }%
% -- Version 2:
% (based on package xparse)
\DeclareDocumentCommand{\newdualentry}{ O{} O{} m m m m }{%
	\newglossaryentry{gls-#3}{%
        name={#5},%\glsadd{#3}
		text={#5},%
		description={#6},%
        crossRef={#3},%
        #1}%
	\newacronym[crossRef={gls-#3},#2]%
        {#3}%
        {#4}%\glsadd{gls-#3}
        {#5}%
}%
% Usage Syntax:
% \newdualentry[glossary options][acronym options]{label}{abbrv}{long}{description}
% and for the references:
% Refer to acronym with \gls{label} and the glossary with \gls{gls-label}
%  -  -  -  -  -  -  -
% Example:
%        \newdualentry%
%        [%
%            deutsch={Programmierschnittstelle},%
%            parent={},%
%            untertitel={}%
%        ]%Options für Glossary
%        [%
%            plural=APIs,%
%            longplural=Programmierschnittstellen%
%        ]%Options für Acronym
%        {api}%label
%        {API}%abbreviation
%        {Application Programming Interface}%long form
%        {%//%
%            Eine \textit{Schnittstelle zur Anwendungsprogrammierung} [...]]%
%        }%description
%        \glsunset{api}%
% Blank Boilerplate:
%\newdualentry%
%[]% Options für Glossary
%[]% Options für Acronym
%{lbl}% label
%{abbr}% abbreviation
%{longf}% long form
%{desc}% description
%________________________________________________________________________
%						Dual Entry Definitions
%VVVVVVVVVVVVVVVVVVVVVVVVVVVVVVVVVVVVVVVVVVVVVVVVVVVVVVVVVVVVVVVVVVVVVVVV