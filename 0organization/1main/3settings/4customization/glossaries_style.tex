%________________________________________________________________________
%------------------------------------------------------------------------
%							Setups für Glossaries
%/\/\/\/\/\/\/\/\/\/\/\/\/\/\/\/\/\/\/\/\/\/\/\/\/\/\/\/\/\/\/\/\/\/\/\/\
% \defglsentryfmt[\acronymtype]{
%   \ifglsused{\glslabel}
%     {\glsgenentryfmt}% Wenn verwendet normales format
%     {\textit{\glsgenentryfmt}}% beim ersten mal kursiv
% }
\newcommand*{\glsprintmitdeutsch}{%
	{\glsentrylong{\glslabel} (\glsentryname{\glslabel}, dt.:
	\glsentrydeutsch{gls-\glslabel})}%
}%
\newcommand*{\glsprintmitenglisch}{%
	{\glsentrylong{\glslabel} (\glsentryname{\glslabel}, engl.:
	\glsentryenglisch{gls-\glslabel})}%
}%
\newcommand*{\glsprintmitdualentry}{%
  \ifglsused{\glslabel}%
    {%
    	\glsgenentryfmt%
    }{%
    	\ifglshasfield{deutsch}{gls-\glslabel}{%
    		\glsprintmitdeutsch%
    	}{%
	    	\ifglshasfield{englisch}{gls-\glslabel}{%
	    		\glsprintmitenglisch%
	    	}{%
	    		{\glsgenentryfmt}%
	    	}%
    	}%
    }%
}%
\newcommand*{\glsprintohnedeutsch}{%
  \ifglsused{\glslabel}%
    {\glsgenentryfmt}% Wenn verwendet normales format
    {\glsgenentryfmt}% beim ersten mal anders formatiert
}%
% Alternativen für erste Formatierung:
% textsf, textit, texttt
%
\defglsentryfmt[\acronymtype]{%
	\ifglsentryexists{gls-\glslabel}{%
		\glsprintmitdualentry%
	}{%
		\glsprintohnedeutsch%
	}%
}%
%
\renewcommand*{\glsclearpage}{}%
%
\newlist{myglossarylist}{description}{10}%
\setlist[myglossarylist]{%
	align=left,%
% 	itemindent=0em,%
% 	labelindent=0em,%
% 	listparindent=0em,%
}%
\setlist[myglossarylist,2]{%
	align=left,%
 	leftmargin=2em,%
	itemindent=-2em,%
}%
\newglossarystyle{list+url}
{% based on list style (adapt as required)
  \setglossarystyle{list}%
    \renewcommand{\glossentry}[2]{%
    \item[\glsentryitem{##1}%
          \glstarget{##1}{\glossentryname{##1}}]
       \glossentrydesc{##1}\glspostdescription\space##2%
    \ifglshasfield{url}{##1}{\glspar
     \glsletentryfield{\thisurl}{##1}{url}%
     \expandafter\url\expandafter{\thisurl}}{}}%
}%
\newglossarystyle{list+Deutsch+Untertitel+alternativ+url}{%
	\renewenvironment{theglossary}%
		{\begin{myglossarylist}}{\end{myglossarylist}}%
	\renewcommand*{\glossaryheader}{}%
	\renewcommand*{\glsgroupheading}[1]{}%
	\renewcommand*{\glossentry}[2]{%
		\item[\glsentryitem{##1}%
			  \glstarget{##1}{\glossentryname{##1}}]%
%
		\ifglshasfield{untertitel}{##1}{\normalfont%
			\textbf{(}\glsentryuntertitel{##1}\textbf{)}.\space}{}%
		\ifglshasfield{alternativ}{##1}{\normalfont%
			Auch genannt: \textbf{\glsentryalternativ{##1}}.\space}{}%
		\ifglshasfield{deutsch}{##1}{Zu Deutsch: %
			\textbf{\glsentrydeutsch{##1}}.\space}{}%
		\glossentrydesc{##1}\glspostdescription\space ##2%
    \ifglshasfield{url}{##1}{\glspar
		\url{\glsentryurl{##1}}}{}%
	}%
	\renewcommand*{\subglossentry}[3]{%
		\begin{myglossarylist}%
			\item[\glssubentryitem{##2}%
			\glstarget{##2}{\strut\glossentryname{##2}%
			}]%
%
			\ifglshasfield{untertitel}{##2}{\normalfont%
				\textbf{(}\glsentryuntertitel{##2}\textbf{)}.\space}{}%
			\ifglshasfield{alternativ}{##2}{\normalfont%
				Auch genannt: \textbf{\glsentryalternativ{##2}}.\space}{}%
			\ifglshasfield{deutsch}{##2}{Zu Deutsch: %
				\textbf{\glsentrydeutsch{##2}}.\space}{}%
			\glossentrydesc{##2}\glspostdescription\space ##3.%
		\end{myglossarylist}%
	}%
	\renewcommand*{\glsgroupskip}{\ifglsnogroupskip\else\indexspace\fi}%
}%
\setglossarystyle{list}%
%
% \setacronymstyle{long-short}
%/\/\/\/\/\/\/\/\/\/\/\/\/\/\/\/\/\/\/\/\/\/\/\/\/\/\/\/\/\/\/\/\/\/\/\/\
%							Glossaries fertig
%------------------------------------------------------------------------
%________________________________________________________________________