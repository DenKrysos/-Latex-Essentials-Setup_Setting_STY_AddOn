%________________________________________________________________________
%------------------------------------------------------------------------
%               File Split Notice
%/\/\/\/\/\/\/\/\/\/\/\/\/\/\/\/\/\/\/\/\/\/\/\/\/\/\/\/\/\/\/\/\/\/\/\/\
%   More inside "./0organization/1main/" in the File "settings_inDocument_perDoc.tex"
%       (Unique to every single Project)
%/\/\/\/\/\/\/\/\/\/\/\/\/\/\/\/\/\/\/\/\/\/\/\/\/\/\/\/\/\/\/\/\/\/\/\/\
%------------------------------------------------------------------------
%________________________________________________________________________
%
%
%
%
%
%________________________________________________________________________
%------------------------------------------------------------------------
%							Setups für enumitem
%					Lists - (itemize, enumerate, description)
%/\/\/\/\/\/\/\/\/\/\/\/\/\/\/\/\/\/\/\/\/\/\/\/\/\/\/\/\/\/\/\/\/\/\/\/\
%  More in the "_perDoc.tex" File inside "./0organization/1main/"
%    -> The "_perDoc.tex" comes after the "_basic.tex" and hence its Settings overwrite.
%
%---------------------------
% %  Additional new List-Style, "Enumeration-Inline-Roman" for enumerating with lower-case roman letters inline
\newlist{enuminlrom}{enumerate*}{1}%
\setlist[enuminlrom]{label=\textit{(\roman*)}}%
% - - - - - - - -
% % Additional List-Style, "Enumeration-Paragraph-Roman" for enumeration with upper-case roman letters, without leftmargin, i.e. looks and behaves like a paragraph, but with enumerated label
\newlist{enumparrom}{enumerate}{1}%
\setlist[enumparrom]{%
    label=\textit{\textbf{(}\Roman*\textbf{)}},%
    %wide,%
    align=left,%
    labelindent=0.5\parindent,%
    listparindent=\parindent,%
    leftmargin=0pt,%
    labelwidth=!,%
    itemindent=!,%
    topsep=0ex,%
    itemsep=0ex,%
    parsep=0ex,%
}%
%
%---------------------------
% %  List-Style Settings
% - - - - - - - -
%\setlist[1]{\labelindent=\parindent} % < Usually a good idea
%\setlist[description]{font=\sffamily\bfseries} % the Default
\setlist[description]{%
    topsep=0ex,%
    itemsep=0ex,%
    parsep=0ex,%
    labelindent=0pt,%
    leftmargin=0pt%
}%
\setlist[itemize]{%
    topsep=0ex,%
    itemsep=0ex,%
    parsep=0ex,%
    leftmargin=1em%
}%
\setlist[enumerate]{%
    topsep=0ex,%
    itemsep=0ex,%
    parsep=0ex,%
    leftmargin=1.5em%
}%
% \renewcommand{\descriptionlabel}[1]{\hspace{\labelsep}\normalfont{\color{DenKrColor_DescriptionLabel}\textbf{#1}}:}%
\renewcommand{\descriptionlabel}[1]{\hspace{\labelsep}\normalfont\textbf{#1}:}%
% \renewcommand{\descriptionlabel}[1]{\hspace{\labelsep}\normalfont\textit{#1}:}%
%
%
% - - Templates
%       For Templates, have a look into the corresponding File inside the "8templates" Directory
%/\/\/\/\/\/\/\/\/\/\/\/\/\/\/\/\/\/\/\/\/\/\/\/\/\/\/\/\/\/\/\/\/\/\/\/\
%							enumitem fertig
%------------------------------------------------------------------------
%________________________________________________________________________
%
%
%
%
%
%________________________________________________________________________
%------------------------------------------------------------------------
%							Setups für Glossaries
%       (Additionally defined cmds/macros are included "pre-Document")
%       (Additional added keys are included in Preamble, with the package)
%/\/\/\/\/\/\/\/\/\/\/\/\/\/\/\/\/\/\/\/\/\/\/\/\/\/\/\/\/\/\/\/\/\/\/\/\
\input{"\DenKrLayoutMainRootDir/3settings/4customization/glossaries_style".tex}%
%/\/\/\/\/\/\/\/\/\/\/\/\/\/\/\/\/\/\/\/\/\/\/\/\/\/\/\/\/\/\/\/\/\/\/\/\
%							Glossaries fertig
%------------------------------------------------------------------------
%________________________________________________________________________